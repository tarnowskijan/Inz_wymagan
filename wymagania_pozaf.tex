\section{Wymagania pozafunkcjonalne}

\subsection{Technologie}
\begin{center}
\begin{tabular}[h]{|p{1.6cm}|p{13.5cm}|}
\hline
ID: & WPF\_1 \\ \hline
Nazwa: & Wykorzystanie systemu przetwarzania danych w pamięci \\ \hline
Priorytet: & Wysoki \\ \hline
Opis: & W celu uzyskania wysokiej dostępności systemu oraz minimalizacji opóźnień przyjmuje się, że SWPFK będzie wykorzystywał system przetwarzania danych w pamięci (In-Memory Computing). Jednym z rekomendowanych rozwiązań jest oprogramowanie Gigaspace XAP. \\
\hline
\end{tabular}
\end{center}

\begin{center}
\begin{tabular}[h]{|p{1.6cm}|p{13.5cm}|}
\hline
ID: & WPF\_2 \\ \hline
Nazwa: & Technologia wykonania modułu serwerowego \\ \hline
Priorytet: & Średni \\ \hline
Opis: & Moduł serwerowy SWPFK powinien zostać wykonany w technologii Java w co najmniej 1.7, w celu zapewnienia wysokiej skalowalności aplikacji. \\
\hline
\end{tabular}
\end{center}

\begin{center}
\begin{tabular}[h]{|p{1.6cm}|p{13.5cm}|}
\hline
ID: & WPF\_3 \\ \hline
Nazwa: & Technologia wykonania interfejsu internetowego \\ \hline
Priorytet: & Średni \\ \hline
Opis: & Interfejs internetowy powinien zostać wykonany w technologii HTML oraz Java, ze względu na doświadczenie posiadane przez dział informatyczny Firmy Kurierskiej w budowie oprogramowania w tych technologiach. \\
\hline
\end{tabular}
\end{center}

\begin{center}
\begin{tabular}[h]{|p{1.6cm}|p{13.5cm}|}
\hline
ID: & WPF\_4 \\ \hline
Nazwa: & Technologia wykonania aplikacji klienckiej \\ \hline
Priorytet: & Średni \\ \hline
Opis: & Aplikacja kliencka zostanie wykonana w technologii Microsoft .NET Framework w wersji co najmniej 3.5. Technologia ta zapewni pracę aplikacji na komputerach PC pracujących pod kontrolą systemu operacyjnego Microsoft Windows oraz szerokie możliwości budowania graficznego interfejsu użytkownika. \\
\hline
\end{tabular}
\end{center}

\subsection{Wydajność}
\begin{center}
\begin{tabular}[h]{|p{1.6cm}|p{13.5cm}|}
\hline
ID: & WPF\_5 \\ \hline
Nazwa: & Wydajność usług modułu serwerowego \\ \hline
Priorytet: & Wysoki \\ \hline
Opis: & Czas wykonywania dowolnej usług w module serwerowym nie może być większy niż 300 milisekund. Czas ten jest mierzony od czasu wejścia zapytania do serwera do czasu wysłania odpowiedzi do klienta. Czasy wspomnianych akcji będą odczytywane z logów modułu serwerowego. \\
\hline
\end{tabular}
\end{center}

\begin{center}
\begin{tabular}[h]{|p{1.6cm}|p{13.5cm}|}
\hline
ID: & WPF\_6 \\ \hline
Nazwa: & Wydajność aplikacji klienckiej \\ \hline
Priorytet: & Średni \\ \hline
Opis: & Czas przejścia pomiędzy dowolnymi dwoma ekranami interfejsu aplikacji klienckiej nie może być dłuższy niż 200 milisekund. Czas jest mierzony od wywołania akcji przejścia do załadowania całości wołanego okna. Z pomiaru jest wyłączony ewentualny czas oczekiwania na odebranie odpowiedzi z usługi modułu serwerowego. \\
\hline
\end{tabular}
\end{center}

\begin{center}
\begin{tabular}[h]{|p{1.6cm}|p{13.5cm}|}
\hline
ID: & WPF\_7 \\ \hline
Nazwa: & Czas uruchamiania aplikacji klienckiej \\ \hline
Priorytet: & Niski \\ \hline
Opis: & Sumaryczny czas uruchamiania aplikacji klienckiej nie może być dłuższy niż 30 sekund. Czas ten jest mierzony od wywołania uruchomienia aplikacji do pojawienia się pełnego okna głównego aplikacji, które umożliwi użytkownikowi wykonanie dowolnej akcji. \\
\hline
\end{tabular}
\end{center}

\subsection{Niezawodność}
\begin{center}
\begin{tabular}[h]{|p{1.6cm}|p{13.5cm}|}
\hline
ID: & WPF\_8 \\ \hline
Nazwa: & Niezawodność modułu serwerowego \\ \hline
Priorytet: & Wysoki \\ \hline
Opis: & Moduł serwerowy musi charakteryzować się pełną dostępnością na poziomie 99.995\% czasu w skali miesiąca. Niedostępność modułu serwerowego definiuje się jako brak lub wygenerowanie błędnej odpowiedzi na żądania innych aplikacji i interfejsów SWPFK. Przy czym w tym przypadku do błędnej odpowiedzi nie zalicza się błędów powstałych na skutek niepoprawnej implementacji procesu biznesowego.  \\
\hline
\end{tabular}
\end{center}

\subsection{Wykorzystanie zasobów}
\begin{center}
\begin{tabular}[h]{|p{1.6cm}|p{13.5cm}|}
\hline
ID: & WPF\_9 \\ \hline
Nazwa: & Wykorzystanie pamięci RAM przez aplikację kliencką. \\ \hline
Priorytet: & Wysoki \\ \hline
Opis: & Całkowite wykorzystanie pamięci RAM w dowolnym momencie działania aplikacji klienckiej nie może przekroczyć 768 megabajtów. \\
\hline
\end{tabular}
\end{center}

\begin{center}
\begin{tabular}[h]{|p{1.6cm}|p{13.5cm}|}
\hline
ID: & WPF\_10 \\ \hline
Nazwa: & Wykorzystanie pamięci RAM podczas używania interfejsu internetowego. \\ \hline
Priorytet: & Średni \\ \hline
Opis: & Całkowite wykorzystanie pamięci RAM w dowolnym momencie korzystania z interfejsu internetowego SWPFK przez przeglądarkę internetową nie może przekroczyć 150 megabajtów. Ilość pamięci RAM jest mierzona tylko dla zakładki przeglądarki w której otwarty jest interfejs SWPFK. \\
\hline
\end{tabular}
\end{center}