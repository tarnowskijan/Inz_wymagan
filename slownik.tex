\section{Słownik pojęć i terminów}
\subsubsection*{SWPFK}
Skrót od System Wspomagania Pracy Firmy Kurierskiej. Określa całość opracowywanego systemu, na który składają się wszystkie aplikacje oraz urządzenia.

\subsubsection*{Przesyłka}
List lub paczka o określonych wymiarach oraz wadze, która została nadana przez klienta Firmy Kurierskiej w celu dostarczenie jej do określonego adresata.

\subsubsection*{Kurier}
Pracownik firmy kurierskiej zajmujący się odbieraniem oraz dostarczaniem paczek klientom.

\subsubsection*{POK}
Skrót od Punkt Obsługi Klienta, stacjonarny punkt w którym można skorzystać w niektórych usług firmy kurierskiej.

\subsubsection*{Moduł serwerowy}
Oprogramowanie uruchamiane na wyspecjalizowanym serwerze, który oferuje wysoką moc obliczeniową. Udostępnia on szereg usług z których korzystają pozostałe moduły systemu. Implementuje większość logiki biznesowej systemu.

\subsubsection*{Aplikacja kliencka}
Oprogramowanie przeznaczone do uruchomienia na komputerze klasy PC. Komunikuje się z modułem serwerowym i korzysta z jego usług za pomocą jednego z protokołów sieci Internet. Zapewnia wizualizację danych oraz możliwość ich wprowadzania do systemu.

\subsubsection*{Aplikacja mobilna}
Oprogramowanie uruchamiane na urządzeniu mobilnym, np. telefon z systemem Android, wyspecjalizowany terminal graficzny. Komunikuje się z modułem serwerowym za pomocą sieci Internet i korzysta z jego usług. Wizualizuje dane oraz zapewnia możliwość ich wprowadzania do systemu.

\subsubsection*{Interfejs internetowy}
Witryna internetowa osiągalna za pomocą przeglądarki internetowej. Wykonany za pomocą technologii HTML. Udostępniony w sieci Internet za pomocą serwera WWW.

\subsubsection*{Konto w SWPFK}
Do konta można się zalogować posiadając unikalną w skali systemu nazwę użytkownika oraz ustawione hasło. Logowanie następuje za pomocą jednego z interfejsów SWPFK. Posiadanie konta w systemie SWPFK pozwala na korzystanie z większej ilości funkcji systemu, takich jak podgląd historii nadanych przesyłek, zobowiązania z tytułu wykorzystanych usług i innych. Konto pozwala na jednoznaczną identyfikację użytkownika w systemie oraz określenie jego uprawnień.

\subsubsection*{Terminal mobilny}
Specjalne urządzenie mobilne z ekranem dotykowym posiadane przez kuriera. Zapewnia on dostęp do SWPFK za pomocą zainstalowanej na nim aplikacji mobilnej. Łączy się z siecią Internet uzyskując dostęp do SWPFK za pomocą sieci komórkowej. Posiada wbudowany skaner kodów kreskowych używany do odczytywania numeru przesyłki przedstawionego za pomocą kodu kreskowego.

\subsubsection*{Pamięć RAM}
Pamięć RAM (Random Access Memory) to pamięć operacyjna komputera PC, w której przechowywane są aktualnie uruchomione programy komputerowe.