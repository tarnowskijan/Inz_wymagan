\section{Charakterystyka interfejsów}

\subsection{Interfejs użytkownika}
Graficzny interfejs użytkownika każdego modułu SWPFK powinien być jak najbardziej przyjazny użytkownikowi. Kolorystyka interfejsów graficznych powinna być spójna z kolorystyką Firmy Kurierskiej, co oznacza użycie odcieni kolorów brązowego oraz złotego.

\subsubsection{Wymagania interfejsu użytkownika}
\begin{center}
\begin{tabular}[h]{|p{1.6cm}|p{13.5cm}|}
\hline
ID: & INT\_WYM\_1 \\ \hline
Nazwa: & Wielojęzyczność interfejsu użytkownika \\ \hline
Priorytet: & Wysoki \\ \hline
Opis: & Ze względu na to, że Firma Kurierska prowadzi działalność w kilku krajach Europy Środkowej oraz Zachodniej interfejsy graficzne SWPFK muszą obsługiwać języki:
\begin{itemize}
\item polski,
\item angielski,
\item niemiecki,
\item hiszpański,
\item francuski,
\item czeski,
\item słowacki.
\end{itemize} \\
\hline
\end{tabular}
\end{center}

\begin{center}
\begin{tabular}[h]{|p{1.6cm}|p{13.5cm}|}
\hline
ID: & INT\_WYM\_2 \\ \hline
Nazwa: & Ilość widoków obsługujących jedną funkcjonalność \\ \hline
Priorytet: & Niski \\ \hline
Opis: & Wymagane jest aby wykonanie dowolnej akcji w systemie (np. rejestrowanie paczki) nie wymagało od użytkownika korzystania z więcej niż dwóch różnych widoków lub formatek. \\
\hline
\end{tabular}
\end{center}

\begin{center}
\begin{tabular}[h]{|p{1.6cm}|p{13.5cm}|}
\hline
ID: & INT\_WYM\_3 \\ \hline
Nazwa: & Obsługa skrótów klawiaturowych \\ \hline
Priorytet: & Średni \\ \hline
Opis: & W celu usprawnienia obsługi aplikacji klienckiej powinna ona umożliwiać dostęp do najczęściej używanych funkcjonalności za pomocą skrótów klawiaturowych. \\
\hline
\end{tabular}
\end{center}

%\subsection{Interfejsy zewnętrzne}

\subsection{Interfejsy sprzętowe}
Aplikacja kliencka SWPFK będzie wspierać czytniki kodów kreskowych podłączane do komputera PC poprzez port USB w wersji 2.0.

W sortowniach aplikacja kliencka będzie komunikować się z automatycznymi czytnikami kodów kreskowych za pomocą interfejsu Ethernet. Systemy automatycznego pomiaru parametrów przesyłki będą komunikować się z aplikacją kliencką przez interfejs USB 2.0.

\subsection{Interfejsy programistyczne}
SWPFK powinien posiadać elastyczny interfejs (API) pozwalający w przyszłości na łatwą integrację z systemami innych zewnętrznych firm transportowych.

Ponadto SWPFK będzie udostępniał interfejs w formie usługi SOAP, który pozwoli na portalom współpracującym z Firmą Kurierską sprawdzenie statusu przesyłki podając jej numer.

\subsection{Interfejsy komunikacyjne}
Komunikacja pomiędzy modułem serwerowym systemu a aplikacjami klienckimi oraz mobilnymi będzie się odbywać za pomocą kanału komunikacyjnego wspierającego szyfrowanie protokołem TLS w wersji co najmniej 1.1. Minimalna długość klucza służącego do zabezpieczenia połączenia musi wynosić 128 bitów.

SWPFK będzie udostępniał interfejs internetowy który będzie używał do komunikacji protokołu HTTP w przypadku informacji publicznych oraz HTTPS w przypadku informacji poufnych.